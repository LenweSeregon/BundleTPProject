\documentclass[a4paper]{article}
\usepackage[francais]{babel}
\usepackage[utf8]{inputenc}
\usepackage{ucs}
\usepackage{exemple}
\usepackage{graphicx}
\usepackage{MnSymbol,wasysym}
\usepackage{hyperref}


\formation{L3MI}
\date{19 Janvier 2017}
\matiere{Conception Orientée Objet}
\titre{Pendu}

\newcommand\code[1]{\textsf{#1}}
\newcommand\srdjan[1]{{\color{red} #1}}

\begin{document}

\entete

\section{Explication générale}



\subsection{Difficultés}

La principale difficulté lors de ce projet a été la prise en main de Swing. Effectivement, n'ayant pas réalisé beaucoup de travaux durant mon cursus avec Swing, sa prise en main avec la disposition des élèments n'a pas été des plus simples.

Je n'ai pas rencontré de difficulté technique.

\subsection{Graphisme}

Comme expliqué dans les difficultés pour le point graphisque, la présentation du pendu est très simple. Je n'avais pas encore connaissance de l'utilisation des JLabel, j'ai donc utilisé la fonction drawString pour dessiner mon mot mais si il devait y avoir une chose à changer dans l'immédiat dans le projet serait donc de remplacer cette partie par un JLabel qui permet de centrer facilement un texte. On a ainsi à gauche le clavier et le mot qui se découvre, et à droite un bouton "abandonner" et une grand zone de dessin ou se dessine petit à petit (en respectant le niveau de difficulté) le pendu lorsqu'un mauvaise lettre est proposée.


\subsection{Utilisation}

L'utilisation n'a pas besoin d'être décrire, elle est tout à fait basique et les simples interactions possibles de l'utilisateur sont le clavier virtuel représenté en haut à gauche et le bouton abandonner qui affiche le mot. Il y a aussi les JOptionPane qui permettent de choisir la difficulté ou le choix de relancer une partie ou non.

\section{Ajouts}

Tout les ajouts proposés par le TP ont été réalisés.

\end{document}
