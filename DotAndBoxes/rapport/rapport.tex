\documentclass[a4paper]{article}
\usepackage[francais]{babel}
\usepackage[utf8]{inputenc}
\usepackage{ucs}
\usepackage{exemple}
\usepackage{graphicx}
\usepackage{MnSymbol,wasysym}
\usepackage{hyperref}


\formation{L3MI}
\date{19 Janvier 2017}
\matiere{Conception Orientée Objet}
\titre{Dots and boxes}

\newcommand\code[1]{\textsf{#1}}
\newcommand\srdjan[1]{{\color{red} #1}}

\begin{document}

\entete

\section{Explication générale}



\subsection{Difficultés}

La principale difficulté que j'ai rencontré lors de ce projet est l'organisation dans le temps, en effet, avec beaucoup de choses à faire en dehors de la POO, je n'ai pas pu terminer ma dernière IA qui est cependant presque fonctionnel ! Si possible, il faudrait que vous regardiez le code qui permet d'apprécier le travail, dans la classe OptimumAI, l'IA était presque bonne mais provoquer une exception dans un cas particulier lors de l'exécution

\subsection{Graphisme}

Rien a expliquer sur les graphismes, ils sont assez simples et le menu est assez propre et claire pour parler de lui même. 


\subsection{Utilisation}

Pour selectionner une ligne, il faut simplement cliquer dessus avec la souris, si le cliqeu est réalisé sur une bordure appartenant déjà à quelqu'un, il faut continuer jusqu'à cliquer sur une bonne bordure.

L'IA actuelle joue simplement le remplissage des cases, mais celle presque fonctionnelle dans le code mais qui n'est pas effective à l'execution réalisé une prévision de son coup pour toujours garder un score positif par rapport à son adversaire et ainsi gagner la partie. Il manquait environ 1h probablemnt pour finir l'IA

\section{Ajouts}

Pas d'ajouts particulier pour ce tp puisque rien n'a été demandé, à par l'dée de d'IA "optimal" qui n'est pas loin d'etre finie

\end{document}
