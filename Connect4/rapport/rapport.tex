\documentclass[a4paper]{article}
\usepackage[francais]{babel}
\usepackage[utf8]{inputenc}
\usepackage{ucs}
\usepackage{exemple}
\usepackage{graphicx}
\usepackage{MnSymbol,wasysym}
\usepackage{hyperref}


\formation{L3MI}
\date{19 Janvier 2017}
\matiere{Conception Orientée Objet}
\titre{Puissance 4}

\newcommand\code[1]{\textsf{#1}}
\newcommand\srdjan[1]{{\color{red} #1}}

\begin{document}

\entete

\section{Explication générale}

\subsection{Difficultés}

Comme pour tout les tp actuellement, le problème pour celui-ci a été un manque de temps évident pour la réalisation de l'intelligence artificielle. En effet, avec le monopoly et le tetris attack en même temps, il fallait être sur plusieurs front en même temps et je n'ai ainsi pas pu finir mon IA de minmax pour le puissance 4. Celui-ci ne doit pas être loin d'être opérationnel et demande donc à être regardé pour apprécier le travail.

La deuxième difficulté a été l'implémentation en MVC puisque ce tp est le premier qui le demande.

\subsection{Graphisme}

Rien a expliquer sur le graphisme, je n'ai pas spécialement essayé de rendre le jeu le plus beau ou le plus fidèle possible au jeu, celui ci permet simplement d'avoir une interface facile à utiliser et intuitive au niveau du menu de configuration ou encore du panneau d'affichage pendant le jeu qui dit ce qu'il faut faire.


\subsection{Utilisation}

Pour jouer au jeu, l'utilisateur utilise les fléches gauche et droite qui permetttent de déplacer la piéce de référence qui se trouve au dessus de la grille pour savoir dans quelle colonne nous sommes. La touche entrée permet de positionner son pion à la colonne souhaité.

Le mode suicide n'a besoin que d'être activé et permet d'activer les conditions inverses de victoire

\section{Ajouts}

Tous les ajouts au niveau du mode suicide, de la taille variable, etc... on était réalisé.

\end{document}
